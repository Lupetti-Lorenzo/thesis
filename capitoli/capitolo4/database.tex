\section{Database}

QUI manca la parte dove faccio differenza con DynamoDB


\subsection{Bastion Host}

Il bastion host è una macchina virtuale EC2 che funge da punto di accesso controllato per interazioni di basso livello con l’istanza RDS, come creazione di utenti con privilegi limitati per gli altri servizi e creazione di database e tabelle.  Risiede nella Subnet Privata con Egress per potersi collegare dall’esterno.

\vspace{0,3cm}

L'istanza della macchina EC2 è di classe T2 e size Micro, l'istanza più piccola possibile, con versione Linux AmazonLinux2023.

La risorsa viene inizializzata attraverso scripts \texttt{cloud-init} (per approfondire visitare la \href{https://docs.aws.amazon.com/AWSEC2/latest/UserGuide/user-data.html}{documentazione AWS}) per:

\begin{itemize}
    \item  configurazione iniziale, per installare le dipendenze necessarie, come aws-cli. 
    \item fornire scripts utili tipo get-aws-secrets e db-connect
    \item inizializzare crono di auto-shutdown
\end{itemize}

La gestione delle credenziali di accesso alla macchina è eliminata grazie all’autenticazione tramite SSO, che rimuove inoltre la necessità di esporre il server alla rete pubblica aumentando la sicurezza.

Il bastion è progettato per essere una risorsa temporanea, quindi per ottimizzare efficienza e costi si spegnerà automaticamente dopo 30 minuti di inattività.


\subsection{Database}

Per la gestione dei dati, è stata adottata un'istanza RDS, con configurazione: 
\begin{itemize}
    \item \texttt{engine:} PostgreSQL v16.
    \item \texttt{memoria:} 20GB con la possibiltà di scalare automaticamente fino a 105GB. Memoria SSD di tipo \href{https://docs.aws.amazon.com/AmazonRDS/latest/UserGuide/CHAP_Storage.html#Concepts.Storage.GeneralSSD}{GP2}.
    \item \texttt{istanza:} classe T2 e size Micro.
\end{itemize}
L’istanza viene dispiegata nella subnet privata in modalità Multi-AZ, quindi replicata nelle 2 availability zones, per assicurare alta disponibilità. Nel caso di necessità di scalabilità può essere valutata una migrazione a RDS con Aurora Serverless V2.

Per testare durante lo sviluppo, invece di mantenere tunnel tramite bastion (sezione \ref{sub:connessione_db}) e lavorare sul database dell'ambiente di DEV, ho preferito lanciare un istanza Postgres in locale.

\subsubsection{Gestione utenti}
Il database ha un account amministratore, dedicato alle operazioni di manutenzione attraverso il bastion host. L’unico altro utente è assegnato all’API, con privilegi di R/W sulle tabelle. 
Alla creazione dell’istanza, vengono creati anche i segreti contenenti le credenziali per i suddetti utenti, che verranno consumate tramite Secrets Manager.
Di seguito il codice SQL eseguito per creare l'utente applicativo:

\lstinputlisting[
    language=SQL,
    caption={Creazione utente per l'API con R/W sulle tabelle },
    label={lst:db_utente}
]{code/database/create_db_user.sql}

Questa procedura è stata eseguita manualmente successivamente alla creazione del database, acquisendo la password attraverso Secrets Manager. Inoltre, se vengono create delle tabelle, è necessario assegnare anche i relativi permessi, seguendo le istruzioni dalla riga 4 alla riga 14. 

\subsubsection{Connessione all'istanza}
\label{sub:connessione_db}
Per accedere all'istanza del database, utilizzo il bastion host come tunnel verso l'istanza del database nella rete privata. Questo processo permette di mappare l'indirizzo IP e la porta del bastion (che a sua volta si connette all'indirizzo IP e alla porta privata dell'istanza) in locale. A questo scopo, ho sviluppato uno script bash che automatizza il processo di configurazione e avvio del tunnel. Lo script cerca l'istanza EC2 nell'account AWS e ne estrae l'ID e se la macchina è spenta viene svegliata. Successivamente, recupera le credenziali di accesso al database da AWS Secrets Manager e, infine, avvia una sessione con AWS Systems Manager (SSM) utilizzando come target il bastion.

Ecco il codice bash che implementa questo processo:

\lstinputlisting[
    language=Bash,
    caption={Script bash tunnel.sh per connettersi all'istanza del databse},
    label={lst:tunnel}
]{code/database/tunnel.sh}

Per quanto riguarda AWS SSM, è un servizio che consente di eseguire comandi su istanze EC2 in modo sicuro e senza dover effettuare una connessione diretta tramite SSH. Utilizzando SSM, è possibile gestire le risorse EC2 in modo centralizzato, senza dover esporre le istanze direttamente alla rete pubblica. Utilizzando il comando start-session di AWS SSM, è possibile avviare una sessione interattiva con un'istanza EC2 tramite il bastion host, consentendo l'accesso sicuro al database tramite il tunnel stabilito.

\vspace{0,3cm}

Una volta avviato il tunnel, è possibile collegarsi sostanzialmente in 2 modi:
\begin{itemize}
    \item \texttt{Linea di comando:} con un tool come \texttt{psql} posso avviare una sessione interattiva da terminale.
    
    \lstinline[language=Bash]{psql -p 5432 -h 127.0.0.1 -U username -d database_name}

    \item \texttt{Applicativo:} una applicazione desktop con GUI come \texttt{PGAdmin}, che fornisce molte funzioni per visualizzare e interagire con il database.
\end{itemize}


\subsubsection{Gestione schema e Migrazioni}
Per l'evoluzione dello schema del database, è stato scelto \texttt{Prisma}, che fornisce una solida piattaforma per la gestione delle migrazioni dei dati. Grazie a Prisma, è possibile eseguire in modo efficiente l'aggiornamento dello schema del database, garantendo una transizione fluida e sicura tra le diverse versioni.
Le funzionalità di migrazione offerte da Prisma consentono di apportare modifiche agli schemi del database in modo controllato e reversibile, assicurando la coerenza e l'integrità dei dati durante il processo di evoluzione. L'approccio dichiarativo di Prisma facilita la definizione chiara e concisa delle modifiche da apportare, semplificando così la gestione delle versioni e agevolando il mantenimento della base dati nel corso del tempo. Prisma utilizza il linguaggio proprietario con estenzione .prisma per definire  la struttura delle tabelle, le relazioni tra di esse, i tipi di dati e le regole di validazione.

\vspace{0,3cm}

Per utilizzare Prisma nel progetto, ho scaricato tramite NPM come dipendenza di sviluppo il pacchetto \texttt{prisma}, il tool da linea di comando.

Il file principale contenente lo schema è schema.prisma. Dato che avevo già definito lo schema tramite SQL, mi è bastato eseguire \lstinline[language=Bash]{npx prisma db pull} per inizializzare il file con lo schema del database, da cui sono partito per le successive modifiche.

\vspace{0,3cm}

Ecco il codice con la configurazione iniziale e la definizione della tabella certificati:

\lstinputlisting[
    language=Prisma,
    caption={Definizione dello schema del database in Prisma},
    label={lst:schema_prisma}
]{code/database/schema.prisma}


La keyword \texttt{model} è usata per definire le tabelle con le specifiche di ogni campo.
In \texttt{datasource} definisco il provider, nel mio caso postgresql e l'url di connessione al database, che passo tramite variabile di ambiente.
La keyword \texttt{generator} indica cosa deve essere generato al cambiamento dello schema, al lancio del comando \lstinline[language=Bash]{npx prisma generate}. Sono definiti 2 generatori: client per prisma-client e kysely (quest'ultimo discusso nella sezione \ref{Kysely}).  

\vspace{0,3cm}

\label{prisma_client}
Inoltre, con lo schema Prisma è possibile generare un client da utilizzare nel codice, con la libreria \texttt{prisma-client}. Questo client, chiamato ORM (Object Relational Mapping), offre strumenti per interagire con il database in modo type-safe e astratto tramite un oggetto, eliminando la necessità di scrivere codice SQL manualmente. Tuttavia, offre anche la flessibilità di eseguire query SQL personalizzate se necessario.

Tramite la libreria \texttt{prisma-client}, scaricata come dipendenza di sviluppo e non inclusa nella produzione, vengono eseguite principalmente operazioni di seeding attraverso il comando \lstinline[language=Bash]{npx prisma db seed}. Questo comando esegue il codice definito nel file `seed.ts` situato nella cartella `prisma`, utilizzando il client Prisma per interagire con il database. Prima di utilizzare il client Prisma, è necessario eseguire la generazione per sincronizzare l'ORM con lo schema definito. Di seguito è riportato un esempio di seeding che aggiunge un record nella tabella dei Certificati:

\lstinputlisting[
    language=TypeScript,
    caption={Esempio di seeding},
    label={lst:seeding}
]{code/database/seed.ts}

Da questo esempio possiamo vedere come viene astratta l'interazione con le tabelle del database attraverso l'ORM, chiamando direttamente la creazione sull'oggetto certificati.

\vspace{0,3cm}

Per mantenere l'evoluzione dello schema del database allineata con le modifiche apportate allo schema Prisma, eseguo una migrazione sul database locale utilizzando il comando \lstinline[language=Bash]{npx prisma migrate dev --name v1.x.x}. Questo comando genera un file di migrazione nella directory "migration" sotto "prisma", il quale contiene i comandi SQL necessari per adeguare lo schema del database alle modifiche apportate allo schema.prisma rispetto alla migrazione precedente. È importante sottolineare che questi file di migrazione rappresentano semplicemente una serie di istruzioni SQL che vengono eseguite per aggiornare lo schema del database.

Una volta generato il file di migrazione, procedo con il tunneling verso il database remoto e sostituisco l'URL di connessione nell'ambiente con le credenziali aggiornate. Successivamente, applico la migrazione remota utilizzando il comando \lstinline[language=Bash]{npx prisma migrate deploy}. Quest'ultimo passaggio è fondamentale per sincronizzare le migrazioni con il database remoto, assicurandoci che tutte le modifiche apportate allo schema siano effettivamente applicate.

Il processo che segue la creazione della migrazione locale può essere automatizzato all'interno di un pipeline, il quale esegue automaticamente il comando finale senza richiedere l'avvio manuale del tunnel o il recupero delle credenziali da parte dello sviluppatore. Tuttavia, questa problematica è discussa più approfonditamente nella sezione \ref{evoluzione_db_problematica}.
