
\chapter{Introduzione} 

La presente relazione documenta l’esperienza di tirocinio svoltasi presso l’azienda TAI Software Solutions nel periodo da dicembre 2023 a marzo 2024. L’azienda è specializzata...

Il progetto proposto dall'azienda mira a sviluppare un Sistema di Gestione della Formazione aziendale, per ottimizzare la pianificazione e il monitoraggio delle attività formative all'interno dell'organizzazione. L'obiettivo è garantire la conformità alle normative ISO9001 e ISO27001, fornendo al contempo un ambiente intuitivo e efficiente per gli utenti coinvolti nel processo di formazione.
Questo progetto è stato ideato da un membro del dipartimento delle Risorse Umane dell'azienda, come risposta alla necessità di adattarsi alle normative ISO sopra menzionate, dato l'esito dell'ultima revisione ISO effettuata in azienda, in cui è emersa una mancanza di conformità nell'ambito del monitoraggio delle competenze aziendali.

Durante il tirocinio, ho avuto l'opportunità di collaborare strettamente con un altro tirocinante. Insieme, abbiamo lavorato nella fase di analisi del progetto, discutendo e definendo i requisiti e gli obiettivi. Successivamente, abbiamo suddiviso i compiti in base alle nostre competenze e interessi. Personalmente, mi sono concentrato sull'infrastruttura e sullo sviluppo del backend del sistema, mentre il mio collega ha assunto la responsabilità del frontend.
Durante tutto il processo di implementazione, abbiamo mantenuto una comunicazione costante e efficace. Ci siamo coordinati regolarmente per discutere i progressi compiuti, condividere idee e risolvere eventuali problemi incontrati lungo il percorso. Questo approccio ci ha permesso di lavorare in modo efficiente e di garantire che il progetto avanzasse secondo i tempi stabiliti.

% Durante questo periodo, ho avuto la opportunità di immergermi nelle dinamiche operative dell’azienda, interagendo con professionisti del settore. Ho lavorato e assistito in prima persona all’intero ciclo del software, permettendomi di acquisire una visione globale e completa di come viene normalmente e efficientemente svilupato un progetto software. Ho lavorato e tecnologie che non avevo mai utilizzato prima d’ora, e mi sono interfacciato con altri progetti di grandi dimensioni creati dall’azienda. 
% In questa relazione verranno approfonditi gli obiettivi specifici raggiunti, le scelte effettuate, le sfide affrontate durante l’implementazione e le lezioni apprese durante questa esperienza.

\section{Struttura della tesi}

Il progetto è stato avviato con una fase di preparazione, durante la quale ho partecipato a un corso fornito dall'azienda sui fondamenti di AWS e sui suoi principali servizi. Durante questo periodo, ho anche eseguito alcune prove pratiche per chiarire eventuali dubbi implementativi.

Successivamente, ho avuto un incontro con l'ideatore del progetto per comprendere appieno i requisiti e le aspettative legate al sistema. Questo colloquio mi ha permesso di acquisire una visione chiara del dominio di applicazione e delle esigenze degli utenti finali. 

Successivamente, io e il mio collaboratore abbiamo lavorato alla redazione di un documento di analisi dettagliato, che comprendeva sia i requisiti funzionali che quelli non funzionali, oltre ai mockup del sistema. Questa fase è stata fondamentale per delineare chiaramente gli obiettivi del progetto e per stabilire una solida base per il lavoro successivo. Durante questo processo, abbiamo sottoposto il documento a varie revisioni con le persone coinvolte nel progetto, compresi i nostri tutor aziendali e altri membri del team. Queste revisioni ci hanno permesso di raffinare il problema e di giungere a soluzioni che soddisfacevano le esigenze di tutti gli interessati.

Infine, sono passato all'implementazione del sistema. Durante questa fase, ho dovuto prendere diverse decisioni implementative e scegliere le tecnologie più adatte alle esigenze del progetto. Questo processo è stato guidato dal documento di analisi precedentemente redatto, che ha funzionato da guida per lo sviluppo delle funzionalità del sistema.

Questo ordine si riflette nella struttura della relazione, la quale è organizzata come segue:
\begin{description}
    \item[Capitolo 2] che riporta le informazioni necessarie per comprendere i capitoli seguenti relativamente alle tecnologie utilizzate durante lo sviluppo del progetto di tesi.
    \item[Capitolo 3] che discute la progettazione logica del sistema.
    \item[Capitolo 4] che descrive come è avvenuta l'implementazione del sistema.
    \item[Capitolo 5] che identifica possibili sviluppi futuri e miglioramenti. 
    \item[Capitolo 6] conclusioni.
\end{description}

